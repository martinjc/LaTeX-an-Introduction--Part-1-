\documentclass[mathserif]{beamer}

\usepackage{listings}
\usepackage{showexpl}
\usepackage{xfrac}
\usepackage{dtklogos}
\usepackage{amsmath}
\usepackage{amsfonts}
\usepackage[T1]{fontenc}

%\usepackage{handoutwithnotes}

%\pgfpagesuselayout{3 on 1 with notes}[a4paper,border shrink=5mm]

%\pgfpageslogicalpageoptions{1}{border code=\pgfusepath{stroke}}
%\pgfpageslogicalpageoptions{2}{border code=\pgfusepath{stroke}}
%\pgfpageslogicalpageoptions{3}{border code=\pgfusepath{stroke}}


\lstdefinestyle{latexsty}{
	language={[LaTeX]TeX},
    basicstyle=\small\ttfamily,
    breaklines=true,
    breakindent=0pt, 
    backgroundcolor=\color{lightgray},
    numbers=none, numberstyle=\tiny, stepnumber=1, numbersep=5pt,
    commentstyle=\color{red},
    showstringspaces=false,
    keywordstyle=\color{blue}\bfseries,
    morekeywords={align,begin},
    tabsize=2,
    pos=b
}

\usetheme{default}
\useoutertheme{infolines}
\usecolortheme[RGB={166,5,20}]{structure}
%\setbeamertemplate{items}[circle]
\setbeamertemplate{blocks}[rounded][shadow=false]
\setbeamertemplate{navigation symbols}{}


\title{Introduction to \LaTeX\ - Part 1}
\subtitle{University Graduate College Training Course}
\author[Martin Chorley]{Dr Martin Chorley}
\institute[COMSC]{School of Computer Science \& Informatics, Cardiff University}
\date[10/02/14]{February 10th, 2014}


\begin{document}

	

%--------------- slide -------------------
\begin{frame}{Maths - Introduction}

\vfill
Typesetting maths is one of the major strengths of \LaTeX, and one of the main reasons for its popularity within the scientific community.
\vfill
The \texttt{amsmath} package is frequently used to add to the built in maths typesetting capabilities of \LaTeX -- all the discussion here will assume it has been used.
\vfill
\end{frame}

%--------------- slide -------------------
\begin{frame}{Maths Environments}

\vfill
In order to include maths within our documents, we must tell \LaTeX\ that we intend to place maths within our text. 
\vfill
There are special environments specifically for typesetting maths. For most, there are also shorthands for beginning and ending these environments.
\vfill
Which environment we use depends on whether we want to display the maths inline (so within our text), or display it as a block separate to the text. 
\vfill
If we display maths as a separate block, we can have the block numbered or non-numbered, depending on whether we want to refer to it with the text.
\vfill
\end{frame}

%--------------- slide -------------------
\begin{frame}[fragile]
\frametitle{Inline Maths}
\vfill
To display maths inline with our text we use the \texttt{math} environment:
\vfill
	\begin{LTXexample}[style=latexsty]
		Inline maths is displayed within our text, so we can add equations (such as \begin{math} a^3 + 6b \end{math}) as part of our sentences.
	\end{LTXexample}
\vfill
Typing \texttt{{\textbackslash}begin\{math\} \ldots {\textbackslash}end\{math\}} could become tedious if there are a lot of separate math equations to display. Fortunately there are two shorthands for this environment: \texttt{{\textbackslash}( \ldots {\textbackslash})} and \texttt{\$ \ldots \$}.
\vfill
\end{frame}


%--------------- slide -------------------
\begin{frame}[fragile]
\frametitle{Displayed Equations (without numbering)}
\vfill
To display maths separately from our text (without numbering the equation) we use the \texttt{displaymath} environment, or the \texttt{equation*} environment:
\vfill
	\begin{LTXexample}[style=latexsty]
		Display maths is displayed separately from the text, so we can add distinct equation blocks.
		\begin{displaymath}
		    a^3 + 6b
		\end{displaymath}
	\end{LTXexample}
\vfill
Again, there are shorthands for starting and ending the \texttt{displaymath} environment: \texttt{{\textbackslash}[ \ldots {\textbackslash}]} and \texttt{\$\$ \ldots \$\$}.
\vfill
\end{frame}

%--------------- slide -------------------
\begin{frame}[fragile]
\frametitle{Displayed Equations (with numbering)}
\vfill
To display maths separately from our text while numbering the equation we use the \texttt{equation} environment:
\vfill
	\begin{LTXexample}[style=latexsty]
		Display maths is displayed separately from the text, so we can add distinct equation blocks.
		\begin{equation}
		    a^3 + 6b
		\end{equation}
	\end{LTXexample}
\vfill
There are no shorthands for starting and ending the \texttt{equation} environment.
\vfill
\end{frame}

%--------------- slide -------------------
\begin{frame}
\frametitle{Text mode vs. Math mode}
\vfill
When typing text normally we are in \emph{text mode} -- when typing within a maths environment we are said to be in \emph{math mode}.
\vfill
There are some important distinctions between text mode and math mode. When in math mode:
\vfill
\begin{itemize}
	\item Spaces and line breaks do not have any meaning. Explicit spaces are either derived from the mathematical expression or specified manually
	\item Empty lines are not allowed.
	\item All text is considered to be variable names, and so each letter is typeset as a mathematical variable. To use `normal' text within math mode, special commands must be used.
\end{itemize}
\vfill
\end{frame}

%--------------- slide -------------------
\begin{frame}[fragile]
\frametitle{Symbols}
\vfill
The basic mathematical symbols that can be entered from the keyboard can be used in math mode:
\vfill
	\begin{LTXexample}[style=latexsty]
		$ + - = ! / ( ) [ ] < > | ' : $
	\end{LTXexample}
\vfill
For other symbols, you will need to know the specific \LaTeX\ command required.
\end{frame}

%--------------- slide -------------------
\begin{frame}[fragile]
\frametitle{Greek Letters}
\vfill
Greek Letters are easy to type in math mode - you only need to type the name of the letter required. If the first letter of the name is uppercase, the symbol will be uppercase. If the first letter is lowercase, the symbol will be lowercase.
\vfill
\begin{center}
\begin{tabular}{r | l}
	Command & Character \\
	\hline
	\texttt{{\textbackslash}alpha} & $\alpha$ \\
	\texttt{{\textbackslash}beta} & $\beta$ \\
	\texttt{{\textbackslash}gamma} & $\gamma$ \\
	\texttt{{\textbackslash}Gamma} & $\Gamma$ \\
	\texttt{{\textbackslash}pi} & $\pi$ \\
	\texttt{{\textbackslash}Pi} & $\Pi$ \\
	\texttt{{\textbackslash}phi} & $\phi$ \\
	\texttt{{\textbackslash}Phi} & $\Phi$ \\
	\ldots & \ldots \\				
\end{tabular}
\end{center}
\vfill
\end{frame}

%--------------- slide -------------------
\begin{frame}[fragile]
\frametitle{Greek Letters - variants}
\vfill
Some Greek letters (lowercase epsilon, theta, phi, pi, rho and sigma) have variants, accessed by adding `var' before the letter name.
\vfill
\begin{center}
\begin{tabular}{r | l}
	Command & Character \\
	\hline
	\texttt{{\textbackslash}epsilon} & $\epsilon$ \\
	\texttt{{\textbackslash}varepsilon} & $\varepsilon$ \\
	\texttt{{\textbackslash}theta} & $\theta$ \\
	\texttt{{\textbackslash}vartheta} & $\vartheta$ \\
	\texttt{{\textbackslash}phi} & $\phi$ \\
	\texttt{{\textbackslash}varphi} & $\varphi$ \\
	\texttt{{\textbackslash}pi} & $\pi$ \\
	\texttt{{\textbackslash}varpi} & $\varpi$ \\
	\texttt{{\textbackslash}rho} & $\rho$ \\
	\texttt{{\textbackslash}varrho} & $\varrho$ \\
	\texttt{{\textbackslash}sigma} & $\sigma$ \\
	\texttt{{\textbackslash}varsigma} & $\varsigma$ \\			
\end{tabular}
\end{center}
\vfill
\end{frame}

%--------------- slide -------------------
\begin{frame}[fragile]
\frametitle{Powers and Indices}
\vfill
To raise something in math mode (to represent a power for instance), the \texttt{\^} (caret) character is used:
\vfill
	\begin{LTXexample}[style=latexsty]
		\[ k^2 + j^{2n} - r^{k^6} \]
	\end{LTXexample}
\vfill
To add an index, an underscore (\_) is used:
\vfill
	\begin{LTXexample}[style=latexsty]
		\[ i_k + j_{2n} - r_{k_6} \]
	\end{LTXexample}
\vfill
When multiple terms are raised or lowered they are grouped with \{\}.
\vfill
\end{frame}


%--------------- slide -------------------
\begin{frame}[fragile]
\frametitle{Operators}
\vfill
\LaTeX\ includes many commands defining operators (functions written as words):
\vfill
	\begin{LTXexample}[style=latexsty]
		\[ \sin (2\theta) \]
	\end{LTXexample}
\vfill
\LaTeX will automatically place subscripts under the operator if required:
\vfill
	\begin{LTXexample}[style=latexsty]
		\[ \lim_{x \to \infty} \exp(-x) = 0 \]
	\end{LTXexample}
\vfill
\end{frame}

%--------------- slide -------------------
\begin{frame}[fragile]
\frametitle{Roots}
\vfill
Square roots can be displayed using \texttt{{\textbackslash}sqrt}. The $n^{th}$ root can be shown using \texttt{{\textbackslash}sqrt[n]}. \LaTeX\ will determine the size of the root automatically. Again, to include multiple terms within the root, group them together with \{\}.
\vfill
	\begin{LTXexample}[style=latexsty]
		\[ \sqrt{2} \] 
		\[ \sqrt{x^2 + y} \] 
		\[ \sqrt[4]{6x - 3y} \]
	\end{LTXexample}
\vfill
\end{frame}

%--------------- slide -------------------
\begin{frame}[fragile]
\frametitle{Fractions}
\vfill
The \texttt{{\textbackslash}frac\{numerator\}\{denominator\}} command allows for including fractions in your equations. 
\vfill
	\begin{LTXexample}[style=latexsty]
		\[ \frac{n!}{k!(n-k)!} \]
	\end{LTXexample}
\vfill
Fractions can be embedded inside other fractions:
\vfill
	\begin{LTXexample}[style=latexsty]
		\[ \frac{\frac{1}{x} + \frac{1}{y}}{y-z} \]
	\end{LTXexample}
\vfill
\end{frame}

%--------------- slide -------------------
\begin{frame}[fragile]
\frametitle{Simple Fractions}
\vfill
For simple fractions it may be easier or nicer to use subscripts and superscripts to create a fraction:
\vfill
	\begin{LTXexample}[style=latexsty]
		\[ ^4/_9 \]
	\end{LTXexample}
\vfill
The \texttt{xfrac} package provides the \texttt{{\textbackslash}sfrac} command to create slanted fractions like this:
\vfill
	\begin{LTXexample}[style=latexsty]
		\[ \sfrac{1}{x} \]
	\end{LTXexample}
\vfill
\end{frame}

%--------------- slide -------------------
\begin{frame}[fragile]
\frametitle{Sums}
\vfill
Sums are typeset using the \texttt{{\textbackslash}sum} command:
\vfill
	\begin{LTXexample}[style=latexsty]
		\[ \sum_{i=1}^{10} z_i \]
	\end{LTXexample}
\vfill
\end{frame}

%--------------- slide -------------------
\begin{frame}[fragile]
\frametitle{Integrals}
\vfill
Integrals are typeset using the \texttt{{\textbackslash}int} command:
\vfill
	\begin{LTXexample}[style=latexsty]
		\[ \int_0^\infty \]
	\end{LTXexample}
\vfill
Integration variables are shown with a normal upright d (obtained with the \texttt{{\textbackslash}mathrm} command), along with a small space (obtained with the \texttt{{\textbackslash},} command).
\vfill
	\begin{LTXexample}[style=latexsty]
		\[ \int_0^\infty x^2\,\mathrm{d}x\]
	\end{LTXexample}
\vfill
\end{frame}

%--------------- slide -------------------
\begin{frame}[fragile]
\frametitle{Other ``big'' symbols}
\vfill
There are many other commands that act similarly to \texttt{{\textbackslash}sum} and \texttt{{\textbackslash}int}:
\vfill
\begin{center}
\begin{tabular}{r | l | r | l | r | l}
	\texttt{{\textbackslash}sum} & $\sum$ & \texttt{{\textbackslash}prod} & $\prod$ & \texttt{{\textbackslash}coprod} & $\coprod$ \\
	\texttt{{\textbackslash}bigoplus} & $\bigoplus$ & \texttt{{\textbackslash}bigotimes} & $\bigotimes$ & \texttt{{\textbackslash}bigodot} & $\bigodot$  \\
	\texttt{{\textbackslash}bigcup} & $\bigcup$ & \texttt{{\textbackslash}bigcap} & $\bigcap$ & \texttt{{\textbackslash}biguplus} & $\biguplus$  \\
	\texttt{{\textbackslash}bigsqcup} & $\bigsqcup$ & \texttt{{\textbackslash}bigvee} & $\bigvee$ & \texttt{{\textbackslash}bigwedge} & $\bigwedge$  \\
	\texttt{{\textbackslash}int} & $\int$ & \texttt{{\textbackslash}oint} & $\oint$ & \texttt{{\textbackslash}iint} & $\iint$  \\
	\texttt{{\textbackslash}iiint} & $\iiint$ & \texttt{{\textbackslash}iiiint} & $\iiiint$ & \texttt{{\textbackslash}idotsint} & $\idotsint$  \\
\end{tabular}
\end{center}
\vfill
\end{frame}

%--------------- slide -------------------
\begin{frame}[fragile]
\frametitle{Brackets, braces and delimiters}
\vfill
There are many delimiters and brackets available in \LaTeX:
\vfill
\begin{center}
\begin{tabular}{r | l | r | l }
	\texttt{( a )} & $( a )$ & \texttt{[ b ]} & $[b]$ \\
	\texttt{\{ c \}} & $\{c\}$ & \texttt{| d |} & $|d|$ \\
	\texttt{{\textbackslash}| e {\textbackslash}|} & $\| e \|$ & \texttt{{\textbackslash}langle f {\textbackslash}rangle} & $\langle f \rangle$ \\
	\texttt{{\textbackslash}lfloor g {\textbackslash}rfloor} & $\lfloor g \rfloor$ & \texttt{{\textbackslash}lceil h {\textbackslash}rceil} & $\lceil h \rceil$ \\
	\texttt{{\textbackslash}ulcorner i {\textbackslash}urcorner} & $\ulcorner e \urcorner$ &  & \\

\end{tabular}
\end{center}
\vfill
\end{frame}

%--------------- slide -------------------
\begin{frame}[fragile]
\frametitle{Automatic sizing}
\vfill
Very often the size of parts of an equation will need to change; \LaTeX\ provides the \texttt{{\textbackslash}left}, \texttt{{\textbackslash}right} and \texttt{{\textbackslash}middle} commands to do just this:
\vfill
	\begin{LTXexample}[style=latexsty]
		\[ \left(\frac{x^2}{y^3}\right) \]
	\end{LTXexample}
\vfill
	\begin{LTXexample}[style=latexsty]
		\[ \left\{\frac{x^2}{y^3}\right\} \]
	\end{LTXexample}
\vfill
\end{frame}

%--------------- slide -------------------
\begin{frame}[fragile]
\frametitle{Manual sizing}
\vfill
If you want manual control over the sizing, you can use the \texttt{{\textbackslash}big}, \texttt{{\textbackslash}Big}, \texttt{{\textbackslash}bigg}, and \texttt{{\textbackslash}Bigg} commands.
\vfill
	\begin{LTXexample}[style=latexsty]
		\[ ( \big( \Big( \bigg( \Bigg( \]
	\end{LTXexample}
\vfill
\end{frame}

%--------------- slide -------------------
\begin{frame}[fragile]
\frametitle{Simple Matrices}
\vfill
The \texttt{{\textbackslash}matrix} environment allows simple matrices to be displayed. The syntax is very similar to the syntax used to create tables:
\vfill
	\begin{LTXexample}[style=latexsty]
		\[ \begin{matrix}
		   a & b & c \\
		   d & e & f \\
		   g & h & i 
		\end{matrix} \]
	\end{LTXexample}
\vfill
\end{frame}

%--------------- slide -------------------
\begin{frame}[fragile]
\frametitle{Matrices}
\vfill
Matrices usually have some form of delimiter. We could add these using the \texttt{{\textbackslash}left} and\texttt{{\textbackslash}right} commands, there are predefined environments including delimiters:
\vfill
\begin{center}
\begin{tabular}{r | l | }
	\texttt{pmatrix} & $(\quad )$ \\
	\texttt{bmatrix} & $[\quad ]$  \\
	\texttt{Bmatrix} & $\{\quad \}$  \\
	\texttt{vmatrix} & $|\quad |$  \\
	\texttt{Vmatrix} & $||\quad ||$ 
\end{tabular}
\end{center}
\vfill
\end{frame}

%--------------- slide -------------------
\begin{frame}[fragile]
\frametitle{Matrices}
\vfill
To fill columns and rows of a matrix with dots, use the \texttt{{\textbackslash}cdots}, \texttt{{\textbackslash}vdots} and \texttt{{\textbackslash}ddots} commands:
\vfill
	\begin{LTXexample}[style=latexsty]
		\begin{equation}
		 \begin{pmatrix}
		   a_{1,1} & a_{1,2} & \cdots & a_{1,n} \\
		   a_{2,1} & a_{2,2} & \cdots & a_{2,n} \\
		   \vdots & \vdots & \ddots & \vdots \\
		   a_{m,1} & a_{m,2} & \cdots & a_{m,n}
		\end{pmatrix}
		\end{equation}
	\end{LTXexample}
\vfill
\end{frame}

%--------------- slide -------------------
\begin{frame}[fragile]
\frametitle{Text in equations}
\vfill
As previously mentioned, \LaTeX\ will treat all text within math mode as part of an equation, and typeset each letter as if it were a variable. The \texttt{\textbackslash{text}} command allows you to enter text within math mode:
\vfill
	\begin{LTXexample}[style=latexsty]
		\[ 50 \text{ apples}  \]
	\end{LTXexample}
\vfill
Similarly, the text formatting commands introduced previously can be used:
\vfill
	\begin{LTXexample}[style=latexsty]
		\[ 50 \textbf{ apples} + 30 \textrm{ pears} \]
	\end{LTXexample}
\vfill
\end{frame}

%--------------- slide -------------------
\begin{frame}[fragile]
\frametitle{Spacing in math mode}
\vfill
There are multiple commands for inserting spaces in equations, they each insert a slightly different amount of space:
\vfill
\begin{center}
\begin{tabular}{r | l | }
	\texttt{{\textbackslash},} & $(\,)$ \\
	\texttt{{\textbackslash}:} & $(\:)$ \\
	\texttt{{\textbackslash};} & $(\;)$ \\
	\texttt{{\textbackslash} } & $(\ )$ \\
	\texttt{{\textbackslash}quad} & $(\quad)$ \\
	\texttt{{\textbackslash}qquad} & $(\qquad)$ \\
\end{tabular}
\end{center}
\vfill
\end{frame}

%--------------- slide -------------------
\begin{frame}[fragile]
\frametitle{Math fonts}
\vfill
Using the \texttt{amsfonts} package provides access to a number of fonts for formatting text within mathematical equations:
\vfill
\begin{center}
\begin{tabular}{r | l | }
	\texttt{{\textbackslash}mathnormal} & $\mathnormal{ABCDEFabcdef123456}$ \\
	\texttt{{\textbackslash}mathrm} & $\mathrm{ABCDEFabcdef123456}$ \\
	\texttt{{\textbackslash}mathit} & $\mathit{ABCDEFabcdef123456}$ \\
	\texttt{{\textbackslash}mathbf} & $\mathbf{ABCDEFabcdef123456}$ \\
	\texttt{{\textbackslash}mathsf} & $\mathsf{ABCDEFabcdef123456}$ \\
	\texttt{{\textbackslash}mathtt} & $\mathtt{ABCDEFabcdef123456}$ \\
	\texttt{{\textbackslash}mathcal} & $\mathcal{ABCDEFabcdef123456}$ \\
	\texttt{{\textbackslash}mathfrak} & $\mathfrak{ABCDEFabcdef123456}$ \\
	\texttt{{\textbackslash}mathbb} & $\mathbb{ABCDEFabcdef123456}$ \\		
\end{tabular}
\end{center}
\vfill
\end{frame}

%--------------- slide -------------------
\begin{frame}[fragile]
\frametitle{Equation Numbering}
\vfill
As already discussed, the \texttt{equation} environment automatically numbers your equations. We can also label them with the \texttt{{\textbackslash}label} command so that we can refer to them in text later.
\vfill
	\begin{LTXexample}[style=latexsty]
		\begin{equation}
		\label{eq:myequation}
		    f(x) = (x+a)(x+b)
		\end{equation}
		We can refer to our equation~\ref{eq:myequation}, or even use~\eqref{eq:myequation} to get a different style of reference.
	\end{LTXexample}
\vfill
\end{frame}

%--------------- slide -------------------
\begin{frame}[fragile]
\frametitle{Cases}
\vfill
The \texttt{cases} environment allows piecewise functions:
\vfill
	\begin{LTXexample}[style=latexsty]
		\begin{equation}
		    |x| = 
		    \begin{cases}
		        -x & \text{if}\ x < 0, \\
		        0 & \text{if}\ x = 0, \\
		        x & \text{if}\ x > 0.
		    \end{cases}
		\end{equation}
	\end{LTXexample}
\vfill
\end{frame}

%--------------- slide -------------------
\begin{frame}[fragile]
\frametitle{Maths in \LaTeX}
\vfill
This is only the basics of what can be accomplished with \LaTeX\ when working with mathematical equations.
\vfill
There are many more packages that can be useful for this type of scientific work. Just some are:
\vfill
\begin{center}
\begin{tabular}{r | p{0.6\textwidth} | }
	\texttt{IEEEtrantools} & includes the \texttt{{\textbackslash}IEEEeqnarray} command for aligning equations nicely\\
	\texttt{mathtools} & includes \texttt{amsmath} and adds extra settings, symbols and environments \\		
	\texttt{mchem} & for typesetting chemical symbols and equations \\		
\end{tabular}
\end{center}
\vfill
\end{frame}


%--------------- slide -------------------
\begin{frame}
\frametitle{Exercise 5}

\begin{center}
\vfill
Use the mathematics formatting commands you've learnt to practice typesetting equations
\vfill
\end{center}
\end{frame}

\end{document}